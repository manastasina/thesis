\clearpage
\section{Results and discussion}

	\subsection{Conservative redisues within NS1 regulate dsDNA binding and control transcription of cellular genes (I)}
	
		The nucleic acid binding function of NS1 is extremely important for influenza A replication and residues R38 and K41 very conservative across viral subtypes \parencite{Hatada1992, Zohari2008}. The mutations R38A, K41A were shown to prevent dsRNA binding by NS1 and result in increased IFN production by infected cells and severely attenuated viral phenotype \parencite{Donelan2003}. Although the proposed role of nucleic acid binding by NS1 is sequestration of viral replication intermediates away from cellular \gls{PRR}s, it is largely arguable as discussed in the Review of Literature. To address the nucleic acid binding function of NS1 using the recombinant influenza A/WSN/33 (WSN) viruses expressing \gls{wt} NS1 or R38A, K41A (RK/AA) mutant NS1. In agreement with previous studies \parencite{Min2006}, RK/AA WSN virus exhibited severely attenuated phenotype in cell culture. 
		
		Gene expression profiling of human macrophages infected RK/AA viruses resulted in strong up-regulation of a large set of cellular genes i(\textbf{I}, fig. 1A and B). Gene set enrichment analysis (GSEA, Broad Institute) indicated that these genes were involved in regulation of innate and adaptive immune responses and belonged to TLR and Jak/STAT signaling pathways, cytokine-cytokine receptor interaction and Natural Killer cell mediated cytotoxicity pathways. Accordingly, RK/AA infection resulted in increased production of both antiviral and pro-inflammatory cytokines (\textbf{I}, fig. 1C). Importantly, the effect of RK/AA mutations on cellular gene expression and cytokine production was not specific to immune cells and was also observed in retinal pigment epithelium (RPE) cell line (\textbf{I}, suppl. fig. 1A and B). 
		
		\Gls{wt} NS1 is associated with cellular chromatin, presumably via protein-DNA, protein-protein or both interactions (\textbf{I}, 2B). Although the direct binding of NS1 to dsDNA has not been shown previously, the structure of NS1 RBD and full-length dimer does not exclude its interaction with dsDNA \parencite{Bornholdt2008, Cheng2009}. Indeed, purified NS1 was interacting with dsDNA with the \gls{Kd} of 11.1~$\pm$~0.7~ $\mu$M, indicative of a likely unspecific binding. The residues R38 and K41 were absolutely required for dsDNA interaction (\textbf{I}, fig. 3B, C). 
		
		The ability of NS1 to associate with cellular chromatin and to bind dsDNA prompted us to test its effects on transcription. To avoid overlapping of direct effects of NS1-dsDNA binding with upstream events this question was addressed in cell-free run-off transcription assay using purified NS1 proteins. Binding of NS1 to DNA had inhibitory effects on DNA-dependent RNA synthesis and time of addition experiment suggested that NS1 inhibits transcription at its initiation step (\textbf{I}, fig. 3D, E). Interestingly, NS1 prevented RNA synthesis after preincubation either with dsDNA or with transcription factors and RNA polymerase II (\textbf{I}, fig. 3F). This finding indicates that NS1-dsDNA interaction inhibits initiation of transcription either by preventing pre-initiation complex loading or by coordinating inhibitory protein-protein interactions of NS1.
		
		\Gls{ChIP} in infected cells revealed presence of NS1 on to the promoter and exon regions of \textit{IFNB1} gene that is strongly induced upon influenza A infection. Interaction to \textit{IFNB1} DNA was R38-, K41-dependent, suggesting direct binding observed in out \textit{in vitro} assay (\textbf{I}, fig. 4). The pattern of NS1 association with \textit{IFNB1} promoter and exon regions was opposite to that of RNA polymerase II, suggesting possible inhibition of \textit{IFNB1} gene expression due to NS1 DNA-binding. 
		
		Functional versatility of NS1 complicates addressing the role of NS1 R38 and K41 in the context of viral infection, as in addition to nucleic acid binding these residues are involved in inhibitory interactions with TRIM25, RIPLET and \gls{RIG-I}, which could also prevent transcriptional activation of \textit{IFNB1} \parencite{Gack2009, Rajsbaum2012}. To split NS1 transcriptional effects from its role in virus-induced \gls{RIG-I} signaling, we addressed the function of its R38 and K41 in transfected cells overexpressing NS1. In such conditions, transcription of \textit{IFN} and \gls{ISG}s can be induced by exogenous exposure to polyinosinic:polycytidylic acid (poly(I:C)), which is detected by cell surface \gls{TLR}3 \parencite{Karpala2005}. As expected, NS1 capable of dsDNA binding strongly inhibited poly(I:C)-induced activation of \textit{IFNB1}, \textit{IFNA16} and \textit{IFNA1}, supporting our hypothesis of its role in direct transcriptional inhibition (\textbf{I}, fig. 5C).
		
		In summary, these results provide a novel insight into IFN responses control by NS1, suggesting direct binding of NS1 to promoter and exon regions DNA and prevention of effective transcription of cellular genes. The high \gls{Kd} that we observed suggests that NS1 binding to dsDNA may also require its oligomerization. This could be possible, as NS1 concentration in cellular nucleus is very high \parencite{Marazzi2012} and it has been previously shown to oligomerize \textit{in vitro} around dsRNA \parencite{Bornholdt2008}. Interestingly, the tunnel diameter inside the tubular NS1 oligomer was reported to be 20 \gls{A} \parencite{Bornholdt2008}, and is more likely to accommodate a 20 \gls{A} B-form DNA double helix, rather than 26 \gls{A} RNA double helix. With regards to binding specificity, our results suggest the lack of it, however, one can speculate that during infection transcriptionally active DNA, e.g. \textit{IFN} genes, is more accessible to NS1, and can thus warrant certain level of quasi-specificity of NS1 binding. This question, however, needs to be addressed with genome-wide studies, for example, \gls{ChIP} combined with sequencing, which, however, is complicated by the lack of \gls{ChIP}-grade anti-NS1 antibodies. Further studies in \gls{RIG-I}-deficient systems would be helpful for dissecting the NS1 functions in pre-transcriptional and transcriptional control of interferon responses. 								
						
	\subsection{RNA-binding domain of NS1 enhances protein synthesis  (II, IV)}
	
		The control of effective viral protein synthesis in the context of infection is an essential function of \gls{NS1} which secures viral replication. NS1 regulates general protein synthesis by preventing  \gls{PKR} activation and its residues 38, 41 and 123-127 are thought to be essential for this \parencite{Lu1995, Min2007}. In addition, its interaction with translation initiation factors \gls{eIF4GI} and \gls{PABP}I and an assumption that NS1 specifically binds viral mRNAs were proposed to warrant recruitment of these factors for specific enhancement of viral protein synthesis \parencite{DelaLuna1995, Aragon2000, Burgui2003}. 
		
		To separate effects of NS1 on translation from its functions in other processes we addressed regulation of protein synthesis in \gls{RRL} cell-free translation system. To identify possible subtype-specific effects we used purified recombinant NS1 proteins of human avian highly pathogenic H5N1, avian low pathogenic H5N2 and pH1N1/2009 strains. 
		
		H5N1 NS1 enhanced translation of reporter mRNA and protein synthesis by \gls{RRL} over 6 fold (\textbf{II}, fig. 2A and C). Importantly, the RNA-binding function of NS1 was not required for translation regulation in \gls{RRL} and RK/AA mutant NS1 was as efficient enhancer of protein synthesis as \gls{wt} (\textbf{II}, fig. 2A). This observation allowed us use RK/AA NS1 proteins in further experiments as they show remarkably better solubility compared to the \gls{wt} \parencite{Bornholdt2008}. Both viral H5N1 and H5N2 NS1s were efficient translation enhancers in \gls{RRL}, whereas the pH1N1/2009 NS1 did not affect protein synthesis (\textbf{II}, fig. 2A and C). 
		
		We addressed the specificity of translational enhancement by NS1 by programming the \gls{RRL} with RNAs extracted from infected cells at different time points post infection. In such system NS1 did not display any specificity towards viral RNAs and promoted translation of both viral and cellular RNAs (\textbf{IV}, fig. 1A). This finding was not surprising due to our observation that NS1 RNA-binding was not required for its translational control in \gls{RRL}. 
		
		In addition to its interactions with translation factors, NS1 has been shown to associate with \gls{hStau} on polysomes in infected cells \parencite{Falcon1999}. As \gls{hStau} has been implicated in regulation of mRNA stability, we addressed whether translational enhancement in RRL in presence of \gls{NS1} was a result of increased mRNA stability. This appeared not to be the case and NS1 had no effect on the rate of mRNA degradation (\textbf{II}, fig. 2E and F). Instead, NS1 enhanced association of mRNA with ribosomes and polysomes in RRL reactions which suggested its possible role in translation initiation (\textbf{IV}, fig. 1B).
		
		To map NS1 residues NS1 essential for translation stimulation we utilized differential effects of H5N1, H5N2 and pH1N1/2009 NS1 proteins on RRL protein synthesis. The NS1 protein of pH1N1/2009 contains an 11 C-terminal truncation, but neither deletion of these residues nor deletion of the whole ED affected capability of avian NS1s to stimulate translation and  RBD alone was sufficient for this function (\textbf{II}, fig. 2G and H). The RBD of pH1N1/2009 NS1 contained several unique amino acids different from H5N1 and H5N2: N25, G26,N48 and W67 (\textbf{II}, fig. 1A). Introduction of this residues into H5N1 NS1 resulted in its loss of control over of RRL protein synthesis. A reciprocal experiment when Q25, N26, S48 and R67 from H5N1 NS1 were introduced in pH1N1/2009 NS1 resulted in its gain of function confirming that these residues are essential for translational control in RRL (\textbf{II}, fig. 2I). The side chains of these amino acids are exposed to solvent, indicating their possible involvement in interaction between NS1 and its protein partner (\textbf{II}, fig. 4B).
		
		NS1 enhanced translation in RRL by preventing \gls{eIF2a} phosphorylation (data not shown). There are four kinases that can phosphorylate \gls{eIF2a} in response to different stress signals: \gls{PKR}, \gls{PERK}, \gls{GCN2} and \gls{HRI} \parencite{Donnelly2013}. Of these, only \gls{PKR} has been reported to act during influenza infection \parencite{Garcia2006}. However, the RBD of A/Udorn/72 H3N2 NS1 which is 88 \% identical to RBD of H5N1 NS1 and contains amino acids Q25, E26, S48, K67 was shown to be insufficient of \gls{PKR} inhibition \parencite{Min2007}. These observations together with our results allow to speculate that \gls{NS1} of some influenza A subtypes evolved to secure protein synthesis during cellular stress by preventing \gls{eIF2a} phosphorylation by PKR and another \gls{eIF2a} kinase. The exact protein, with which NS1 interacts in this process and biological relevance of such interaction, however, are yet to be identified.	In summary, our data indicates an extra level of control over general translation by NS1. As NS1 proteins used in this study lacked RNA binding activity, our results can not exclude additional specific regulation of viral protein synthesis in which RNA-binding may be involved.
		
	\subsection{C-terminus of NS1 contributes to modulation of host antiviral responses (III)}
		
		Despite NS1 is a well structured protein, its C-terminus is likely to be unstructured \parencite{Hale2008b}. Nevertheless, it occurs to play an essential regulatory role in infection, as it accommodates numerous interaction motifs and modification sites that can modulate virus-host interactions, ofter in strain-specific manner \parencite{Liu2010, Marazzi2012, Li2001, Melen2012, Hsiang2012, Hsiang2009}. Natually occurring truncations and extensions of NS1 C-terminus are not uncommon \parencite{Suarez1998}. The NS1 protein of H1N1 IAVs isolated in 2009-2014 is typically 219 aa long, H1N1 viruses encoding NS1 of 202 and 230 aa have been recently reported in Finnish patients \parencite{Lakspere2014}. However, the relevance of NS1 C-terminal truncations and extensions to viral replication and virulence is unclear.
		
		We aimed to address the functional role of NS1 C-terminus length using a influenza A viruses of well-characterized A/WSN/33 background that express NS1 proteins of 202, 220 and 230 amino acids. These viruses were used to infect human macrophages.
		
		Sequential truncations of NS1 C-terminus increased transcriptional activation cellular genes and the virus expressing shortest NS1 was least successful in control of host gene expression (\textbf{III}, 2A). The majorty of genes, differentially regulated by NS1 C-terminus length, e.g. \textit{MX1}, \textit{ISG15}, \textit{OASL}, \textit{CCL8}, \textit{IL6}, and \textit{IL8}, was involved in the regulation of innate and adaptive immune responses (\textbf{III}, fig. 2B).  This effect was to some extent observed on protein level and amounts of IL-6 and IL-8 secreted by macrophages were increased upon NS1 truncations (\textbf{III}, fig. 2C and D). 
		
		Differential regulation of immune-related gene expression and cytokine production prompted us to address the role of NS1 C-terminus in cellular signaling pathways. We observed that C-terminal truncations of NS1 upregulated phosphorylation of several phosphoproteins involved in immune reponses actiation. The shorter NS1 C-terminus was, the stronger phosphorylation of STAT3, JNK, HSP60 and AMPK$\alpha$1 was observed (\textbf{III}, fig. 3). 
		
		As a sequence of its limited capability to counteract development of antiviral responses, the virus expressing 202 aa long NS1 displayed delayed growth kinetics (\textbf{III}, suppl. fig. 1A).
		
		We used the viruses that express 230 and 202 aa long NS1 to address the effect of NS1 C-terminus deletion \textit{in vivo}. NS1 C-terminus truncation reduced influenza A virulence and pathogenicity in mice (\textbf{III}, fig. 4A and B), suggesting the regulatory role of NS1 C-terminus in infection.
		
		C-terminus of A/WSN/33 NS1 contains human-like PDZ-binding motif for which no interactions have yet been identified and a phosphorylation site T215 which effect is also unclear \parencite{Jackson2010, Hsiang2012}. Nevertheless, our data indicates the functional role of NS1 C-terminus length in regulation of host immune responses and in viral pathogenicity and virulence. Altough sporadic truncations and extensions of NS1 are quite seldom events, they could provide opportunities to lose or acquire functional motifs within NS1 C-terminus. Notably, the amino acidic composition and C-terminus length of significantly contribute to NS1 diversity. Moreover, the H1N1 influenz A viruses isolated from 2009 demonstrate highest substitution rate in NS1 compared to other viral subtypers and as long as amino acidic variations in NS1 C-terminus are not detrimental, they provide opportunity for the virus to probe additional ways of virus-host interactions fine tuning \parencite{Xu2011}. 
		 
	
	\subsection{Translational modulation by NS1 provides a tool to improve cell-free protein synthesis system (IV)}
	
		The effects of NS1 on translation in \gls{RRL} observed in Study II prompted us to utilize NS1 for improvement of \gls{RRL} cell-free translation system. \gls{RRL} is widely used for translation of specific templates, membrane protein studies, post-translational events in folding and protein modification, and high-throughput screening for protein-protein interactions \parencite{Fuller2000, Douthwaite2012, Fixsen2010, Wang2011}. However, low protein yield is a considerable limitation of RRL, which should be overcome to improve the current applications \parencite{Carlson2012}. As viruses have evolved multiple ways to facilitate protein synthesis, a possible way to improve cell-free protein synthesis is addition of viral factors which regulate translation. These include viral proteins and specific regulatory structures on mRNA such as \gls{UTR}s or \glspl{IRES}. 
	
		The NS1-mediated enhancement of \gls{RRL} translation occurred through inhibition of \gls{eIF2a} phosphorylation and stabilization of translation initiation and was not specific to viral mRNAs (\textbf{IV}, fig. 1). We reasoned RRL translation could be reinforced using mRNAs with specific 5' and 3' regulatory structures which together with NS1 would exert cumulative effect. For this several reporter mRNAs with \gls{5meG} or \gls{IRES} and with different 3' \gls{UTR}s at the optimal concentration (\textbf{II}, fig. 2D; \textbf{IV}, fig. 2A).
		
		In attempt to We tested several for mRNAs with different 5' and 3' regulatory elements


Usage of mRNAs with specific structures or leader sequences can be another common strategy for increasing efficiency of CFTS. Hence, we next addressed the key mRNA properties allowing the most robust protein production in NS1-supplemented RRL. For this, we evaluated translation of several luciferase-encoding mRNAs with distinct 5’ (7-methylguanosine (m7G) cap or IRES) and 3’ (poly(A) tail or 3’ untranslated region (UTR)) control elements (Fig. 3A). Expression constructs for different reporter mRNAs were cloned in pGL3 vector as described previously (40-42). The corresponding mRNAs were translated in RRL with or without NS1RK/AA and luminescence was used as readout. We observed that although NS1RK/AA enhanced translation of all the mRNAs studied, this enhancement as well as total protein production varied depending on the control elements of the mRNA in question. In particular, we found that hepatitis C virus (HCV) 5’ IRES-containing mRNAs with a 3’ poly(A) tail or HCV 3’UTR were significantly better translated in the presence of NS1RK/AA (4.8 and 4.3 fold increase, respectively), but resulted in lower levels of luminescence, compared to other mRNAs studied (Fig. 3B). Translation of mRNAs containing 5’ m7G cap and human beta-globin leader sequence, and 3’ poly(A) tail or simian vacuolating virus 40 (SV40) UTR in the presence of NS1RK/AA was also enhanced (4.3 and 4.7-fold, respectively), and resulted in higher levels of luminescence, especially in the case of poly(A)-containing mRNA. Finally, translation of encephalomyocarditis virus (EMCV) IRES and poly(A)-containing mRNA appeared to be most robustly regulated by NS1RK/AA (11.2-fold increase) resulting in the highest luminescence levels of all the mRNAs studied. These data suggest that supplementing RRL with NS1RK/AA in combination with mRNA with 5’ EMCV IRES and 3’ poly(A) can represent a beneficial strategy to improve the protein yield of the RRL translation system. 
	
\newpage
\section{Conclusions}

	rk/aa virus as a tool for immunity study and vaccine