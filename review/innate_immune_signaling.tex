Upon sensing influenza A virus,TLR3 induces proinflammatory responses in several ways. First, together with its \gls{TLR} adapter protein TRIF, it can lead to activation of \gls{NFkB} and 
		
		
		Schulz2005
		DC phagocyte infected material, induce signalling through TLR3
		increase presentation to CD8+ CTL
		induce cross-priming (i.e. response induction) of CD8+ CTL
		
		LeGoffic2006
		Pulmonary TLR3 expression is enhanced in vivo during IAV infection
		TLR3-/- animals express reduced proinflammatory cytokines and less CD8+ CTLS
		Higher viral titers, but less inflammation and better survival 
		TLR3-/- excessively produce IFNgamma, that may be a negative regulator of CD8+ T-cell responses (and their number, it can increas apoptosis in T-cells) -- could be a lates protective mechanism to reduce TCL-mediated immunopathology
		
		% % % % % % % % % % % % % %
		TLR7
		% % % % % % % % % % % % % %
			Lund2004
				pDCs
				TLR7 recognized IAV and VSV
				TLR7 is indispensable, TLR7-/- mice don't secrete IFNalpha in response to IAV
				
				88-/- mice don't secrete IFNalpha and IL12 in reposnse to IAV 
				The exact viral ligand for TLR7 is unknown
				TLR7 detects in endosomes
				Still not clear, because by rapidly expressing cytokines DCs rarely get infected by IAVs -- so how can they have virus in endosome? DCs are not good in phagocyting large particles (Stent, Reece etal 2002 Cytometry)
				
				Diebold2004
				in pDCs
				IFNa responses require endosomal recognition of genomic IAV RNA by TLR7 and MyD
				recognition requires endosomal acidification
				MyD88 adaptor is required
				TLR7 is critical receptor (among TLRs), TLR3 and 8 are dispensable
				Induces IFNA, IL6, IL12
				purified IAV RNA induced TLR7 -- its still not clear how it happens during infection
				they think (in discussion) that IAV generates dsRNA -- can use it for the conflict
				%possbile model -- degraged IAV particles in endosomes
				%specificity to IAV RNA also unknown, TLR7 can be induced by self-RNA too
				
				Honda2005
				TLR9 acts via Myd88-IRF3 complex (nothing about IAV though), just that Myd88 is a common adapter for TLR
				
				Sasai2010
				TLR9 is in pDCs (maybe somewhere else too?)
				TLR9 induces antiviral, but not proinflammatory cytokines
				induciton of antiviral cytokines through IRF3
				TLR9 when activated is cleaved 
				Cleaved TLR9 is trafficked to another endosome and through TRAF3 and IRF7 induces NFkB signalling and IFN genes
				AP-3 is required for trafficking from endosome to specialized lysosome-related organelle
				
				Seo2010
				MyD88 is required for recruitment of CD11b+ granulocytes, production of inflammatory cytokines, CD4+ proliferation, Th1 cytokine produciton by T-cells
				TLRs are important for induction of adaptive immune responses
				Responses can be dependent of viral subtype -- MeD88 is not required for H1N1, but is required for H3N2 (Seo2010, LeGoffic2006)
		
		% % % % % % % % % % % % %
		RIG-I
		% % % % % % % % % % % % % %
		
			Yoneyama2005
				Three RLRs: RIG-I, MDA-5, LGP2
				RIG-I and MDA-5 sense vRNA and have overlapping functions, knockdown of either of them partially blocks IFN signaling
				LGP2 is their negative regulator
				RIG-I and MDA-5 are antiviral (supress yield; VSV, EMCV)
				IFN -- positive feedbaack
						
				Seth2005
				MAVS C-terminal transmembrane domain in mitochondria
				MAVS N-terminal CARD functions downstream of RIG-I and upstream of TBK1
				TBK1 and IKKe phosphorylate IRF3 (not their work)
				
				Pichlmair2006
				IAV does not generate dsRNA
				RIG-I can be activated via 5'ppp ssRNA
				
				Siren2006
				rig-i and mda-5 are ifn-inducible
				rig-i and mda-5 induce type I ifns (but TLR3 does not!)
				rig-i and mda-5 regulate antiviral cytokine expression
				for the first time show that rig-i and mda-5 are activated by IAV, although it replicates in nucleus
				
				Gack2007
				TRIM25 E3 ligase
				TRIM25 induces ubiquitination of RIG-I CARDs
				TRIM25 interacts with CARDs of RIG-I, adds Lys63-linked Ub on CARD
				Ub of RIG-I markedly increases its signaling
				
				Cui2008
				RIG-I binds to ss vRNA in 5'ppp dependent manner
				RNA-dependent dimerization activated RIG-I ATPase
				zinc coordination site is essential and conserved amond MDA-5 and LGP2
				structurally binding site in RIG-I is distinct from MDA-5 and LGP2 
				-- defining ligand specificity
				
				Yoneyama2008
				RIG-I -- 2 functions -- RNA recognition and signal induction
				CTD interacts with CARD and represses it
				upon CTD interaction with ligand -- changes condormation, CARD is free and can induce signal
				
				Bamming2009
				role of helicase domain in MDA-5 and RIG-I
			    helicase domains essential for activity and dependent on ATP
			    not strictly required for signal transcduciton
			     
				Myong2009
				DExD/H box helicase domain in ATP-dependent manner translocates on dsRNA
				activation of RIG-I -> translocation on dsRNA
				
				Oshiumi2009
				RIPLET can be promote a Lys-63 ubi independently on TRIM25
				RIPLET has seq similarity to TRIM25
				CTD and helicase domains of RIG-I interact with RIPLET
				RIPLET promotes RIG-I dependent IFNB promoter activation
				
				Zeng2010
				activation of RIG-I requires polyub chains linked to k63 of ub
				unanchored ub can activate RIG-I
				RIG-I binds to unanchored ub chains at k63 on its own
				
				Oshiumi2010
				RIPLET is essential for RIG-I mediated innate responses in vivo
				
				Civril2011
				dsRNA stimulation of RIG-I induces conformational change that results in formation of ATPase site
				
				Bogefors2011
				airway epithelium is the most prominent location of RLRs
				
				Luo2011
				crystal structure of RIG-I-dsRNA
				RIG-I wraps dsRNA
				dimerization and oligomerization of RIG-I is induced by dsRNA and enhances RIG-I activities
				its dependent on dsRNA length
				
				Kowalinski2011
				structure of autorepressed RIG-I
				CARDs are sequestered by helical domain
				dsRNA and ATP biding induce major conformation changes which liberate CARDs for downstream signaling
				
				Oshiumi2013
				Riplet is required for TRIM25-mediated ub of RIG-I
				
				Patel2013
				oligomerization of RIG-I occurs in ATP dependent manner and dsRNA length dependent manner		
				
				LeGoffic2007
				TLR3 resulates proinflammatory response (activates NFkB but not 
				IRF3 dependent gene expression)
				RIG-I and MDA-5 regulate both proinflammatory and antiviral signalling
										
				LeGoffic2007
				RIG-I, MDA5, MAVS are constitutively expressed in epithelial cells
				IAV upregulated RIG-I and MDA5	