\newpage
\section{Review of the literature}

	\subsection{Introduction}
	
	Viruses are seemingly simple in comparison to prokaryotic and eukaryotic cells and to multicellular organisms, comprising only a nucleic acid and a subset of proteins. Because of this simplicity they have very limited capacity to encode factors essential for their own replication and are fully dependent on the host cells. Independently on the virion structure and properties, on the nature of the viral genome and on the replication strategy, viruses have to find suitable ways to interact with the cell at each step of the arbitrary viral life cycle. For this, specific viral proteins interact with the numerous cellular factors and subvert normal cellular processes to fulfill viral needs.
			
	The initial interaction event between the virus with a susceptible cell occurs when the viral binds to the cell surface receptor. Be it an enveloped virus like ...., or a non-enveloped like .... the specific viral proteins interact with cellular factors to allow uncoating~--- the opening of viral capsid and delivery of genetic material to cellular cytoplasm or nucleus. 
		
	Independently on whether they encodes its own RNA polymerase or use the cellular enzyme, viruses set up complex interactions with cellular transcription machinery to ensure effective synthesis of its own mRNA. For example, Poliovirus and \gls{RVFV}, which use their own enzymes for RNA synthesis both shut down host transcription, although in different ways: poliovirus \gls{3cpro} cleaves \gls{TBP}~---~a core factor of cellular transcription~---, whereas \gls{RVFV} interacts with \gls{TFIIB} \parencite{Kundu2005, LeMay2004}. In contrast, the viruses that recruit host transcription machinery set up regulatory interactions with cellular transcription factors to support their own transcription, e.g. \gls{HSV} regulatory protein VP16 interacts with cellular transcription factors Oct-1 and HCF-1, empowering transcription from viral \gls{IE} promoters \parencite{Wysocka2003}.

	Furthermore, being restricted in their encoding capacity viruses lack their own functional translational machinery and even \textit{Pandoravirus salinus} with its biggest known viral genomes of 2.77 \gls{Mb} obtains only few required translation factors remaining fully dependent on the protein synthesis of its Acantamoeba host \parencite{Philippe2013}. Viruses target translational machinery to secure preferential translation of viral transcripts or to shut off host translation when it is not required. For example, \gls{VSV} secures preferential translation of its mRNAs via 3' mRNA structure \parencite{Whitlow2006} and in addition ensures effective protein synthesis with the help of ribosomal protein rpL40 \parencite{Lee2013}. In contrast, \gls{CRPV} sets up an intimate interplay with host translation machinery to empower its remarkable factorless initiation of translation \parencite{Bushell2002}. and the  
	
	alternative ways to initiate translation 
	 
	
	They further use cellular processes for assembly and transport of new virions (more info here). Finally, a number of cellular proteins are involved in viral exit, whether it is accompanied by cell lysis or not (examples here).
	
	The cells are not inert and are able to recognize the presence of the virus quickly and respond with a robust induction of innate immune responses. No successful viral replication would be possible without a proper control of these responses and independently of the specific strategies of their replication viruses encode specific factors that target immune responses, adding an additional level of complexity of virus-host interactions. Specific proteins that target multiple cellular processes to prevent development of antiviral responses are found across all viral families. This adds another level of complexity to virus-host interactions. 

	The number of strategies developed by the viruses is overwhelming. Although we were able to deduce the strategies of viral life cycle from the nature of viral genome already in 1971 \parencite{Baltimore1971}, the exact mechanisms of viral replication are obviously so diverse and complicated up until now we strive to understand them. We study virus-host interactions largely in attempt to fight numerous threats that humankind faces from pathogenic viruses. These involve viral surveillance, development of drugs and vaccines. We also study virus-host interactions as viruses has proven many times to be a great tool to understand cell biology, starting from discovery of DNA replication in .... up to recent .... . Finally, understanding of the mechanisms behind virus-host interactions brings us novel applicable tools widely used in biotech (examples). Although a great deal of mechanisms has been already discovered, the amount information that we get now is growing exponentially and majority of discoveries is perhaps still ahead. 
		
	\subsection{Influenza A virus: an overview}
	
	This work is dedicated to influenza A virus, a member of  \textit{Orthomyxoviridae} family. Influenza A viruses are commonly classified based on their surface antigens \gls{HA} and \gls{NA}. All 16 subtypes of \gls{HA} and all 9 subtypes of \gls{NA} are found in wild birds which, apparently, represent the natural reservoir of influenza A \parencite{Stallknecht2007}. However, certain subtypes of the virus can also infect domesticated birds and multiple species of mammals, including humans. Whereas influenza A virus is asymptomatic in its natural hosts, it can cause mild to severe intestinal infections in poultry and asymptomatic to severe respiratory infections in mammals \parencite{Webster1992a}. 
	
	Influenza A virus genome is composed of a \gls{(-)ssRNA} \parencite{Palese1977}. It is replicated with the viral \gls{rdrp} which is known to be error-prone and is estimated to make $1.5$ and $7.5\times10^5$ misincorporations per nucleotide, meaning that on average one mutation appears the viral genome after each replication cycle \parencite{Drake1993, Parvin1986}. The gradual accumulation of mutations in viral proteins is referred to as antigenic drift.	In addition, the viral genome is segmented: each virion contains eight RNA molecules that encode viral proteins \parencite{McGeoch1976}. The genomic segments can recombine during the co-infection of the same cell with distinct influenza A viruses giving rise to reassortant progeny virions that contain segments derived from both ``parent'' viruses \parencite{Desselberger1978}. Such major changes in the virus are referred to as antigenic shift. Antigenic drift and antigenic shift are the drivers of viral evolution.
		
	%species-specificity defined by sialic acids
	%cross species barriers
	%mix in pigs
	
	
	Although the first human influenza A virus was isolated in 1933 \parencite{Smith1933} and the first confirmed influenza A pandemic occurred the 1918 \parencite{Taubenberger1997}, numerous records indicate that humankind has been facing influenza epidemics and probably also pandemics for at least several centuries \parencite{Potter2001}. The vast majority of human influenza A infections are caused by H1N1 and H3N2 viruses.
	
	Circulating strains of influenza A cause seasonal infections in humans. In most countries these infections result in annual epidemics which may affect up to 10 \% of the population worldwide and result in up to $500000$ deaths (\hyperlink{www.who.in}{World Health Organization}). In addition to these annual epidemics, global pandemics can occur. They result from viral spillover from other species and occur when humans are infected with the viruses to which they are immunologicaly na\"{i}ve. Although influenza A pandemics are relatively rare events, humankind has faced four pandemics in XX century and already two in the XXI century.  Whereas the morbidity and mortality of seasonal influenza is average, the morbidity and mortality of pandemic influenza is unpredictable and can vary from very modest to very high.
	
	Influenza A infects up to 10 \% of human population annually and kills up to $500000$. Importantly, it imposes not only a constant health care threat, but also a significant economic burden to agriculture and health care. Annual human infections cost billions of euros to developed countries and infection of poultry costs XXX. 
	
	The segmented RNA nature of influenza A virus genome results in high mutation rate and confers possibility for recombination. Because of that there is a constant threat of novel pandemics and also existing antivirals and vaccines have very limited efficacy. Therefore, there is a constant effort worldwide to improve the control over influenza. 
	
	Although influenza A is a thoroughly studied virus, its changing nature requires constant improvement of efforts to control influenza spread. These efforts require thorough virus surveillance based on understanding of viral ecology, improvement of existing vaccines and development of effective antivirals which require comprehensive understanding of virus-host interactions.  
	
	Influenza A viruses are enveloped viruses with a pleiomorphic virions sized approximately 30 nm. It has a segmented single-stranded RNA genome of negative polarity. Eight genomic segments of 800--2000 base pairs which, depending on the subtype, encode 11--14 proteins. The virion is formed by a layer of matrix protein 1, surrounded by the membrane derived from the infected cell after budding. In the viral membrane there are three proteins: hemagglutining and neuraminidase~--- important viral glycoproteins~--- and matrix protein 2, which forms the ion channel. Inside the virion there is a viral genome in complex with nucleoprotein and viral polymerase. This is called viral ribonucleoprotein. 
	
	The nature of viral genome allows fast changes which happen due to genetic drift and genetic shift. These changes significantly complicate development of ways to control influenza and require constant improvement of vaccines and antivirals. It has become clear the the promising strategies for that should be based on thorough understanding of virus-host interactions.
	
	
	\textit{I will start with a short introduction to IAV mentioning why people worldwide put so much effort in studying Influenza A viruses (epidemics, pandemics + sporadic outbreaks with pandemic potential, brifely economic costs for health care and agriculture)}
	
	\textit{Briefly discus what is IAV, the organization of the virion and viral genome (don't forget to mention vRNPs -- it is an important link to the chapter discussion detection of the pathogen associated molecular patterns (PAMPS))} \\
	
	 \textit{Mention here genetic drift and genetic shift as drivers of viral evolution and main reasons why we continuously need to improve options to control IAV.} \\

	\subsection{Influenza A virus replication cycle}
	
	\textit{In this section I will give a thorough, yet concise overview of the viral replication cycle.}

	\subsection{Cellular responses to Influenza A infection}
	
	\textit{Here I will first mention that viral infection induces immune responses, which can be innate or adaptive. Proper establishment of innate immune responses is important not just for initial counteraction of viral infection, but also for the onset of the adaptive response.}
		
		\subsubsection{Detection of the virus}
		
		\textit{Start with explaining what are PAMPs and PRRs. Then discuss what are the PAMPS for Influenza, and cellular PRRs for IAV detection. Mention that IAV replication cycle has stages in teh cytoplasm and in the nucleus. Discuss TLRs in the cellular membrane and the endosome, RLRs, NLRs in the cytoplasm.}
		
		\subsubsection{Induction of type I interferons}
		
		\textit{Describe the mechanisms of primary antiviral response establisment and produciton of IFNs.}
		
		\subsubsection{Activation of interferon-stimulated genes}
		
		\textit{Discuss induciton of interferon-stimulated genes and their role in counteracting viral replication: RNASeL/OASL, NOS, PKR.}
		
		\subsubsection{Establishment of cellular antiviral state}
	
		\textit{Describe etablishment of cellular antiviral state and how antiviral state affects key cellular functions preventing virus replication. Possibly (?) discuss intercellulaur signalling and attraction of immune cells to the site of infection --- not sure yet about this, might be too much and confusing.}
		
	\subsection{Viral means to counteract antiviral responses}
	
		\textit{Discuss here involvement of various viral proteins in counteracting development of antiviral responses (mention mask/hit strategies). State the critical role of NS1 and its many functions in teh cell that are necessary to secure viral replication.}
			
	\subsection{NS1 protein}
		
		\subsubsection{Structure of NS1}
		
		\textit{Shortly discuss the structure of NS1, RBD, ED, C-terminus}
		
		\subsubsection{NS1 synthesis, modifications and localization}
			
		\subsubsection{Modulation of host processes by NS1}
		
		\textit{Bring in the detailed information on NS1 interactions with viral and cellular factors with a focus on limitation of cellular responses to infection. Possibly subdivide this section further in subsections. Establish links to translation and transcription studies that are included in the paper.}
		
		\subsubsection{NS1 diversity}
		
		\textit{Discuss natural diversity of NS1. This is an important link to C-terminus manuscript and also to Kainov et al. study. Discuss here also that NS1 is an important virulence determinant, possibly refer to 1918 H1N1 NS1 and H5N1 NS1 proteins.}
		
		\subsubsection{NS1 as a therapeutic target}
		
		\textit{Discuss the importance of NS1 for successful viral replication, possibly (?) delNS1 vaccine and attempts to develop drugs targeting NS1. Finish with stressing the importance of thorough understanding of IAV-host cell intercations for development of novel treatment options and IAV surveilance.}