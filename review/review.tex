\newpage
\setcounter{page}{1}
\section{Review of the literature}

\subsection{Introduction}
	
	Viruses are seemingly simple in comparison to prokaryotic and eukaryotic cells and to multicellular organisms, comprising only a nucleic acid and a subset of proteins. Because of this simplicity they have very limited capacity to encode factors essential for their own replication and are fully dependent on the host cells. Independently on the virion structure and properties, on the nature of the viral genome and on the replication strategy, viruses have to find suitable ways to interact with the cell at each step of the arbitrary viral life cycle. For this, specific viral proteins interact with the numerous cellular factors and subvert normal cellular processes to fulfill viral needs.
			
	The initial interaction event between the virus with a susceptible cell occurs when the viral binds to the cell surface receptor. Be it an enveloped virus like ...., or a non-enveloped like .... the specific viral proteins interact with cellular factors to allow uncoating~--- the opening of viral capsid and delivery of genetic material to cellular cytoplasm or nucleus. 
		
	Independently on whether they encodes its own RNA polymerase or use the cellular enzyme, viruses set up complex interactions with cellular transcription machinery to ensure effective synthesis of its own mRNA. For example, Poliovirus and \gls{RVFV}, which use their own enzymes for RNA synthesis both shut down host transcription, although in different ways: poliovirus \gls{3cpro} cleaves \gls{TBP}~---~a core factor of cellular transcription~---, whereas \gls{RVFV} interacts with \gls{TFIIB} \parencite{Kundu2005, LeMay2004}. In contrast, the viruses that recruit host transcription machinery set up regulatory interactions with cellular transcription factors to support their own transcription, e.g. \gls{HSV} regulatory protein VP16 interacts with cellular transcription factors Oct-1 and HCF-1, empowering transcription from viral \gls{IE} promoters \parencite{Wysocka2003}.

	Furthermore, being restricted in their encoding capacity viruses lack their own functional translational machinery and even \textit{Pandoravirus salinus} with its biggest known viral genomes of 2.77 \gls{Mb} obtains only few required translation factors remaining fully dependent on the protein synthesis of its Acantamoeba host \parencite{Philippe2013}. Viruses target translational machinery to secure preferential translation of viral transcripts or to shut off host translation when it is not required. For example, \gls{VSV} secures preferential translation of its mRNAs via 3' mRNA structure \parencite{Whitlow2006} and in addition ensures effective protein synthesis with the help of ribosomal protein rpL40 \parencite{Lee2013}. In contrast, \gls{CRPV} sets up an intimate interplay with host translation machinery to empower its remarkable factorless initiation of translation \parencite{Bushell2002}. and the  
	
	alternative ways to initiate translation 
	 
	
	They further use cellular processes for assembly and transport of new virions (more info here). Finally, a number of cellular proteins are involved in viral exit, whether it is accompanied by cell lysis or not (examples here).
	
	The cells are not inert and are able to recognize the presence of the virus quickly and respond with a robust induction of innate immune responses. No successful viral replication would be possible without a proper control of these responses and independently of the specific strategies of their replication viruses encode specific factors that target immune responses, adding an additional level of complexity of virus-host interactions. Specific proteins that target multiple cellular processes to prevent development of antiviral responses are found across all viral families. This adds another level of complexity to virus-host interactions. 

	The number of strategies developed by the viruses is overwhelming. Although we were able to deduce the strategies of viral life cycle from the nature of viral genome already in 1971 \parencite{Baltimore1971}, the exact mechanisms of viral replication are obviously so diverse and complicated up until now we strive to understand them. We study virus-host interactions largely in attempt to fight numerous threats that humankind faces from pathogenic viruses. These involve viral surveillance, development of drugs and vaccines. We also study virus-host interactions as viruses has proven many times to be a great tool to understand cell biology, starting from discovery of DNA replication in .... up to recent .... . Finally, understanding of the mechanisms behind virus-host interactions brings us novel applicable tools widely used in biotech (examples). Although a great deal of mechanisms has been already discovered, the amount information that we get now is growing exponentially and majority of discoveries is perhaps still ahead. 
		
\subsection{Influenza A virus: an overview}
	
	This work is dedicated to influenza A virus, a member of  \textit{Orthomyxoviridae} family. Influenza A viruses are commonly classified based on their surface antigens \gls{HA} and \gls{NA}. All 16 subtypes of \gls{HA} and all 9 subtypes of \gls{NA} are found in wild birds which, apparently, represent the natural reservoir of influenza A virus \parencite{Stallknecht2007}. However, certain subtypes of the virus can also infect domesticated birds and multiple species of mammals, including humans. Whereas influenza A virus is asymptomatic in its natural hosts, it can cause mild to severe intestinal infections in poultry and asymptomatic to severe respiratory infections in mammals \parencite{Webster1992a}. 
	
	Influenza A virus genome is composed of a \gls{(-)ssRNA} \parencite{Palese1977}. It is replicated with the viral \gls{rdrp} which is known to be error-prone and is estimated to produce between $1.5$ and $7.5\times10^5$ misincorporations per nucleotide. Because \gls{rdrp} also lacks proofreading activity these misincorporations can not be repaired and on average one mutation appears the viral genome after each replication cycle \parencite{Drake1993, Parvin1986}. The gradual accumulation of mutations in viral proteins is referred to as antigenic drift. In addition, the viral genome is segmented: each virion contains eight RNA molecules that encode viral proteins \parencite{McGeoch1976}. The genomic segments can reassort during the co-infection of the same cell with distinct influenza A viruses giving rise to progeny virions that contain segments derived from both ``parental'' viruses \parencite{Desselberger1978}. Such major changes in the virus are referred to as antigenic shift. Antigenic drift and antigenic shift are the key drivers of viral evolution \parencite{Forrest2010}.
		
	Although the first human influenza A virus was isolated in 1933 \parencite{Smith1933} and the first confirmed influenza A pandemic occurred the 1918 \parencite{Taubenberger1997}, numerous records indicate that humankind has been facing influenza epidemics and probably also pandemics for at least several centuries \parencite{Potter2001}. There is molecular evidence for influenza A \gls{HA} subtypes 1, 2, 3, 5, 7 and 9 can infect humans, the  majority of human influenza A infections are caused by H1 and H3 viruses.
	
	Circulating strains of influenza A cause seasonal infections in humans. In most countries these infections result in annual epidemics which may affect up to 10 \% of the population worldwide and result in up to $500000$ deaths (\hyperlink{www.who.in}{World Health Organization}). 
	
	In addition to these annual epidemics, global pandemics can occur when humans are infected with the viruses to which they are immunologicaly na\"{i}ve. Although influenza A pandemics are relatively rare events, humankind has faced three major pandemics in XX century and already one in the XXI century \parencite{Lagace-Wiens2010, Fineberg2014}.  Whereas the mortality of seasonal influenza is modest, the mortality of pandemic influenza is unpredictable and can vary: for example, the mortality during H1N1 pandemic in 2009 was only below 0.5~\%, but the mortality during the H5N1 pandemic in 1997 was close to 60~\% \parencite{Forrest2010, Noah2013}. In addition, influenza imposes enormous economic burden both to health care and to agriculture \parencite{Szucs1999, Noah2013}. 
	
	Because of the limited antivirals and vaccines efficacy due to antigenic drift and because of a constant risk for new to control influenza A. These efforts require thorough virus surveillance based on understanding of viral ecology, improvement of existing vaccines and development of effective antivirals which require comprehensive understanding of virus-host interactions.  
	
\subsection{Influenza A virus organization and replication cycle}

	Influenza A virions are pleiomorphic, i.e. their shapes are not uniform and can be spherical, kidney- or rod-shaped with the average size of 100--150 nm \parencite{Fujiyoshi1994}. The outer shell of the virions is composed of the host-derived lipid bilayer in which viral \gls{HA}, \gls{NA} and \gls{M2} are incorporated. This shell is underlined with the viral \gls{M1} \parencite{Harris2006}. Each virion encompasses genomic RNA segments packed in specific structures referred to as \gls{vRNP}. They represent supercoiled ring-like structures in which paired 5' and 3' ends of the viral RNA are associated heterotrimetic viral polymerase complex and the rest of the RNA is densely covered with \gls{NP} \parencite{Arranz2012}. In the virion the \gls{vRNP}s are associated with the \gls{M1} protein \parencite{Rees1982, Ye1999}. Eight genes of all influenza A viruses encode 10 essential viral proteins: \gls{HA}, \gls{NA}, \gls{M1}, \gls{M2}, \gls{NP}, \gls{PB1}, \gls{PB2}, \gls{PA}, \gls{NS1}, and \gls{NEP} \parencite{Lamb1983}. In addition, some influenza A strains may encode accessory proteins PB1-F2, PB1-N40, PA-X, PA-155 and PA-182. Whereas \gls{HA}, \gls{NA}, \gls{M1}, \gls{M2}, \gls{NP}, \gls{PB1}, \gls{PB2}, \gls{PA} and \gls{NEP} are structural components of viral particle, \gls{NS1}, PB1-F2, PB1-N40, PA-X, PA-155 and PA-182 are considered to be non-structural and are involved in regulation of virus-host interactions \parencite{Chen2001,Hale2008b,Wise2009,Jagger2012,Muramoto2013}.
	
	The viral replication cycle begins when the viral \gls{HA} binds to the specific virus receptor on cell surface. The key, but possibly not the only receptors for influenza A virus are sialic acids linked to cellular surface glycoproteins or glycolipids \parencite{Skehel2000, Stray2000, Martin1998}. \gls{HA} molecules of avian influenza A viruses recognize $\alpha$-2,3-linked sialic acids and \gls{HA} of human influenza A viruses recognize $\alpha$-2,6-linked sialic acids \parencite{Connor1994, VanRiel2010}. After receptor binding the viruses are endocytosed via clathrin-dependent or clathrin- and caveolin-independent routes and are transferred towards the perinuclear space in endosomes \parencite{Dourmashkin1974, Matlin1981, Sieczkarski2002, Lakadamyali2003}. Acidification of late endosomes in perinuclear space triggers two essential events that allow virus uncoating and delivery of \gls{vRNP}s to the cytoplasm. Firstly, low pH mediates the conformational change in the \gls{HA} enabling fusion of viral and endosomal membranes \parencite{Carr1993}. Secondly, acidification of the virus interior leads to dissociation of \gls{M1} from the \gls{vRNP}s which is required for their successful import in the nucleus \parencite{Bui1996}. 
	
	In the nucleus the \gls{vRNA}s are transcribed \textit{in cis} by the viral polymerase associated with the \gls{vRNP} \parencite{Moeller2012}. Synthesis of viral mRNA is initiated using a 10--13 nucleotide long primers with \gls{5meG} \parencite{Beaton1981, Plotch1981a}. These primers are derived via a process of ``cap-snatching'' during which the \gls{5meG} cap structures on cellular mRNAs are recognized and bound by \gls{PB2} subunit of viral polymerase \parencite{Guilligay2008} and further endonucleolytically cleaved by viral \gls{PB1} and \gls{PA} \parencite{Li2001, Dias2009, Yuan2009}. The resulting primers provide \gls{5meG}-caps for viral transcripts and 3'-OH ends enabling mRNA chain elongation by \gls{PB1} \parencite{Poch1989}. The synthesis of viral mRNA terminates after reiterative copying of 5--7 uridines located at the 5' end of vRNA, resulting in appearance of 150--200 adenine bases at the 3' end of viral mRNA \parencite{Plotch1977, Robertson1981d, Poon1999}. Thus, the viral mRNA is capped and polyadenylated being structurally indistinguishable from cellular transcripts. They are exported via the cellular RNA export machinery to the cytoplasm where they are translated \parencite{Chen2000}. Many of the synthesized viral proteins  shuttle back to the nucleus, where they facilitate production of new \gls{vRNP}s and their export to the cytoplasm \parencite{Greenspan1988, Neumann1997, Huet2010, Wang2013}. 
	
	Replication of influenza A viral genomes occurs through two arbitrary steps. First, \gls{cRNP}s containing positive single-stranded cRNA are produced. Next, these \gls{cRNP}s serve as templates for production of progeny \gls{vRNP} \parencite{Elton2005}. In contrast to mRNA, the synthesis of both cRNA and vRNA is carried out \textit{in trans} by the free viral polymerase available in the nucleus after synthesis and import of new viral proteins \parencite{Jorba2009, Moeller2012}. Moreover, initiation of both cRNA and vRNA synthesis does not require cell- or virus-derived primers and occurs \textit{de novo} resulting in the presence of triphosphates at their 5' ends \parencite{Hay1982, Zhang2010}. The progeny \gls{vRNP}s are assembled in the nucleus and contain exact copies of parental vRNA, a single viral polymerase and multiple copies of \gls{NP}. They can be transcribed and later exported from the nucleus for virion assembly \parencite{Resa-Infante2011}. 
	
	The assembly of the virion and budding occurs at the cellular plasma membrane and requires transport of essential components of progeny virions to the site of assembly. The \gls{vRNP}s in a complex with \gls{M1} and \gls{NEP} are exported from the nucleus via cellular CRM1/exportin-1 pathway and then are transported to the site of budding via cellular microtubules \parencite{Akarsu2003, Momose2007, Kawaguchi2012}. Viral envelope proteins (\gls{HA}, \gls{NA} and \gls{M1}) obtain specific sorting signals for their delivery to the budding site \parencite{Hughey1992, Kundu1996, Tall2003} and are transported there through the Golgi network \parencite{Daniels-Holgate1989}. The virion assembly is localized to specific  cholesterol- and sphingolipid-enriched regions of plasma membrane referred to as lipid rafts \parencite{Scheiffele1999}. The budding requires \gls{M1}, \gls{HA} for initiation of cellular membrane curvature, coordinated interaction of \gls{M1} and \gls{vRNP}s for packaging of viral genomes and \gls{M2} for bud scission \parencite{Nayak2009a, Rossman2011}. Finally, the viral \gls{HA} cleaves the sialic acids are cleaved off the cellular surface releasing new virions that can initiate another infection cycle \parencite{Barman2004}.

	
	\subsection{Host factors involved in IAV replication cycle}
	
	Because of the limited capacity of influenza A virus to encode its own proteins, its effective replication relies on cellular factors. A large number of such factors have been recently identified using yeast two-hybrid assay, genome-wide RNAi screening, and proteomic approaches \parencite{Mayer2007, Brass2009, Shapira2009, Hao2008, Karlas2010, Konig2010, Shaw2011, Song2011}. At least 128 of these factors were identified in two or more screens simultaneously. Functional clustering of these factors revealed their involvement in essentially all stages of viral replication \parencite{Watanabe2010}. 
	
	Thus, the clathrin-mediated endocytosis of influenza A requires cellular clathrin epsin-1 \parencite{Chen2008a} and efficient endosomal transport depends on cellular GTPases Rab5 and Rab7 \parencite{Sieczkarski2003}. Fusion of viral and endosomal membranes is dependent on \gls{vATPase} that acidifies endosomal interior and several subunits of this macromolecular complex have been identified as host factors required for influenza A replication \parencite{Watanabe2010}. 
	
	Nuclear import of \gls{vRNP}s occurs in an active way and requires the interaction of \gls{NP} with importin $\alpha$1 or importin $\alpha$5 \parencite{Cros2005}. Although influenza A RNAs are transcribed by its own \gls{RdRp}, it interacts with cellular DNA-dependent RNA polymerase II presumably to facilitate cap snatching \parencite{Engelhardt2005}. Furthermore, influenza A utilizes cellular splicing machinery to process its mRNAs derived from segments 7 and 8 \parencite{Dubois2014} and cellular nuclear export machinery to deliver its transcripts to the cytoplasm \parencite{York2013}. As the virus encodes none of the translation machinery components, its protein synthesis completely depends on the host translation machinery and the efficacy of viral protein production is secured via the tight interaction between viral \gls{NS1} and cellular translation factors \parencite{DelaLuna1995, Burgui2003, Aragon2000}.
	
	Effective synthesis of \gls{vRNP}s requires interaction of \gls{RdRp} with cellular minichromosome maintainance complex \parencite{Kawaguchi2007}, serine/threonine phosphatase 6 \parencite{York2014}. The export of \gls{vRNP}s is dependent on the interaction of \gls{M1}-\gls{vRNP} with the cellular nuclear export receptor CRM1, which presumably occurs via viral \gls{NEP} \parencite{Brunotte2014} and their further transport to the budding site requires interaction of viral \gls{NP} with cellular Rab11 GTPase \parencite{Eisfeld2011}. Finally, assembly of the virion at the budding site and bud formation requires functional actin microfilaments and cellular energy sources \parencite{Nayak2004}. 
	
	The above mentioned interactions give just few examples of a complex interactome that the virus establishes during infection: accession of host-pathogen interaction database  \parencite{Kumar2010} yielded published associations of viral proteins with over 400 host factors. Importantly, viral replication occurs under cellular antiviral responses and therefore to securing it is not limited to recruiting host factors for essential stages of viral replication cycle, but also expands to counteraction of virus detection by the cell and its downstream events. 
	
	
	\subsection{Host responses to Influenza A infection}
	
	\textit{Here I will first mention that viral infection induces immune responses, which can be innate or adaptive. Proper establishment of innate immune responses is important not just for initial counteraction of viral infection, but also for the onset of the adaptive response.}
		
		\subsubsection{Detection of the virus}
		
		\textit{Start with explaining what are PAMPs and PRRs. Then discuss what are the PAMPS for Influenza, and cellular PRRs for IAV detection. Mention that IAV replication cycle has stages in teh cytoplasm and in the nucleus. Discuss TLRs in the cellular membrane and the endosome, RLRs, NLRs in the cytoplasm.}
		
		\subsubsection{Induction of type I interferons}
		
		\textit{Describe the mechanisms of primary antiviral response establisment and produtioon of IFNs.}
		
		\subsubsection{Activation of interferon-stimulated genes}
		
		\textit{Discuss induciton of interferon-stimulated genes and their role in counteracting viral replication: RNASeL/OASL, NOS, PKR.}
		
		\subsubsection{Establishment of cellular antiviral state}
	
		\textit{Describe etablishment of cellular antiviral state and how antiviral state affects key cellular functions preventing virus replication. Possibly (?) discuss intercellulaur signalling and attraction of immune cells to the site of infection --- not sure yet about this, might be too much and confusing.}
		
	\subsection{Viral means to counteract antiviral responses}
	
		\textit{Discuss here involvement of various viral proteins in counteracting development of antiviral responses (mention mask/hit strategies). State the critical role of NS1 and its many functions in teh cell that are necessary to secure viral replication.}
			
	\subsection{NS1 protein}
		
		\subsubsection{Structure of NS1}
		
		\textit{Shortly discuss the structure of NS1, RBD, ED, C-terminus}
		
		\subsubsection{NS1 synthesis, modifications and localization}
			
		\subsubsection{Modulation of host processes by NS1}
		
		\textit{Bring in the detailed information on NS1 interactions with viral and cellular factors with a focus on limitation of cellular responses to infection. Possibly subdivide this section further in subsections. Establish links to translation and transcription studies that are included in the paper.}
		
		\subsubsection{NS1 diversity}
		
		\textit{Discuss natural diversity of NS1. This is an important link to C-terminus manuscript and also to Kainov et al. study. Discuss here also that NS1 is an important virulence determinant, possibly refer to 1918 H1N1 NS1 and H5N1 NS1 proteins.}
		
		\subsubsection{NS1 as a therapeutic target -- this goes to discussion}
		
		\textit{Discuss the importance of NS1 for successful viral replication, possibly (?) delNS1 vaccine and attempts to develop drugs targeting NS1. Finish with stressing the importance of thorough understanding of IAV-host cell intercations for development of novel treatment options and IAV surveilance.}