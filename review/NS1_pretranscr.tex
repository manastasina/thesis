
		Kochs2007
		some NS1 inhibit IRF3 and IFNB signaling
		strain-specific differences
		
		Haye2009
		in DCs
		TX 1-126 cannot block IFN response
		less late viral protein expression
		Tx 1-99 and 1-126 have more phospho-IRF3
		DCs infected with NS1 mutants have upregulation of IFN I responses, ISGs adn proinflammatory genes. They are better in stimulating adaptive immune responses
		WT human isolate (Tx) is better inhibitor than PR8
		
		Talon2000
		NS1 inhibits IRF3 
		NS1 is a general inhibitor of IFN signaling (plasmid expression)
		
		Opitz2007
		infection of epithelial cells with deficient NS1 promotes RIG-I and MDA5 activation
		RIG-I--MAVS--IRF3 are involved in IFN signaling in response to IAV
		NS1 can inhibit IFN signaling (no mechanism shown)
		
		Mibayashi2007a
		NS1 is in the insoluble (=mt membrane) complex with RIG-I and MAVS and inhibits RIG-I signaling (independent on RNA binding) -- but no direct inhibitory contact shown
		NS1 inhibits RIG-I mediated nuclear translocation of IRF3 and activation of IFNB promoter
		
		Munir2011a
		Allele A NS1 are better in NFkB inhibition than allele B
		this is species independent
		mapped to ED 
		
		Munir2012
		Allele A NS1 are better in inhibition of AP-1 than allele B
		mapped to ED
		
		Geiss2002
		gene expression profiling wt vs delNS1
		NFKBIA, IL-8, TNFRSF6, and HLA-C genes up in delNS1
		actin alpha not affected
		up in delNS1: ISGs, e MX1?MXA, in addition to IFI75, IFI41, IRF1, ISGF3-
		? ?IRF9 and others
		mut in NS1 -- more pronounced expression of IFN and NFkB regulated genes
		STAT pathway stronger activated in del NS1
		NS1 from 1918 is more effective than NS1 WSN
		
		Ludwig2002
		NS1 inhibits JNK and c-Jun
		NS1 inhibits AP-1 regulated genes
		NS1 inhibits dsRNA mediated JNK activation
		R38 K41 needed -> propose that dsRNA binding activity is needed
		both nuclear and cytoplasmic NS1s inhibit
		
		Wang2000
		del NS1 induces NFkB activation and IFNB promoter
		wtNS1 prevents dsRNA- and virus-mediated NFkB induction
		RBD is enough, R38 K41 are important
	
		
		Pauli2008
		5' ppp induces STAT1
		SOCS3 (inhibitor of JAK/STAT) increase in NFkB activated infected cells
		SOCS3 inhibits STAT1
		
		Tisoncik2011
		gene expr with C-term truncated NS1 (1-126)
		stronger induction of IFNB1, CXCL10, IFIT2, IFIT3, MXA -> they claim its because NFkB regulation
		ub-proteasome pathway induced in mutants, TRIM25 only in A549
		IFN-stimulated immune-proteasome and antigen- presentation pathways are regulated by NS1
		
		Ruckle2012a
		non-canonical NFkB is linked to daptive immune responses activation
		NS1 inhibits non-canonical NFkB via RIG-I and inhibits CCL19
		stain-nonspecific
		NS1 inhibits non-canonical NFkB via RIG-I inhibition
		canonical is supressed only partially, non-canonical fully
		
		Gao2012
		NS1 physically interacts with IKK alpha and beta, 
		inhibits IKK beta phosphorylation funtion and prevents deregulation of IkB in classical pathway
		also prevents IKK alpha mediated processing of p100 in p52 in non-canonical pathway
		this inhibits NFkB nuclear translocation
		
		Rajsbaum2012
		avian and mammalian NS1 bind human, but not mouse TRIM25
		in mouse NS1 interacts with RIPLET
		human NS1 binds and inhibits both TRIM and RIPLET
		R38 K41 are important for interaction with Riplet, but maybe via RNA
				
		Gack2009
		NS1 interacts with Trim, blocks rig-i signaling
		E96/97 are important
		R38 K41 are important also
	