NA-binding
		
		Linker
		
		Effector
		
		Tail
		
		Qian1995a --- biochemical characterization
		%RNA-binding domain 1-73 aa
		%RBD can dimerize
		%RBd interacts with RNA
		%Highly alpha-helical, approx. 80 \% is helical
		
		Chien1997 --- crystal structure
		%RNA binding surface is made of antiparallel helices and in arginine rich
		%Structure is six-helical fold, its dimeric; resembles portions of other RNA binding proteins
		Salt bridges buried in the surface are important for protein structure stabilization
		
		%Liu1997a --- Crystal structure
		%monomer --- 3 alpha-helices
		%all helices are (+) charged
		
		Wang1999
		R38 key for RNA binding, K41 is the second
		Dimerization is required for RNA binding
		R38, K41 affect RNA bidning, but not dimerization
		%Helices 2 and 2' contribute to RNA binding (that's where R38 and K41 are)
		%Helices 2 and 2' form a groove within which RNA can lie
		10-fold higher affinity to longer dsRNA (140 bp vs 55 bp)
		
		Chien2003
		NS1 binds dsRNA, but not RNA-DNA duplexes or dsDNA
		Kd 1 uM, stoichiometry 1:1
		helices 2 and 2' are critical
		A-form DNA
		
		Bornholdt2006 --- X-ray PR8
		H5N1 have D92E mutation and deletion of linker (aa 80-84)
		dimer
		monomer is 7 beta strands and 3 alpha helices
		beta strands form antiparallel sheet
		CPSF binding residues are here
		
		Yin2007a --- NMR and X-ray
		%RBD
		%1-73 is RBD
		conservative surface tracks of basic and hydrophilic residues for RNA binding
		
		Bornholdt2008
		gull-length NS1 H5N1
		both domains can participate in inter-ns1 interactions and can form oligomers
		tubular tunnel
		20 angstrom
		cooperatively oligomerizes
		
		Hale2008c
		Trp187 is essential for ED dimerization
		disordered tail
		
		Cheng2009
		crystal structure of NS1 with short RNA
		35,37,38,41 are important
		38 most critical
		several residues outside this patch can facilitate binding
		
		Xia2009
		Udorn ED structure
		
		Kerry2011
		crystal structures of several strains ED
		helix helix interactions in ED are the only conserved parts of ED domain
		helix-helix interaface if variable and transient (bioinformatics)
		dimerization allosterically increases affinity of NS1 to dsRNA
		
		Carrillo2014
		H6N6 full length
		3 ways of ED orientation towards RBD -- open, semi-open, closed
		Preference of these states depend on linker region, residue 71 and a mechanical hinge
		can be autoregulating mechanism that depends on temporal distribution, modifications, localization